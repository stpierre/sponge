\section{Other Solutions}
\label{sec:other-solutions}

There are a vast number of potential solutions to this problem that
would seem to be low-hanging fruit -- far simpler to implement, at
least initially, than our ultimate solution -- but that would not
work, for various reasons.

\subsection{Yum Excludes}

A core yum feature is the ability to exclude certain packages from
updates or installation~\cite{Vid02}.  At first, this would seem to
be a solution to the problem of package versioning: simply install the
package version you want, and then exclude it from further updates.
But this has several issues that made it unsuitable for our use (or,
we believe, this use case in general):

\begin{itemize}

\item It does not (and cannot) guarantee a specific version.  Using
  excludes to set a version depends on that version being installed
  (manually) prior to adding the package to the exclude list.

\item There is no guarantee that the package is still in the
  repository.  Many mainstream repositories\endnote{For instance,
    Extra Packages for Enterprise Linux (EPEL) and the CentOS
    repositories themselves.} do not retain older versions in the same
  repository as current packages.  Consequently, when reinstalling a
  machine where yum excludes have been used to set package versions
  (or when attempting to duplicate such a machine), there is no
  guarantee that the package version expected will even be available.

\item In order to use yum excludes to control package versions, a very
  specific order of events must occur: first, the machine must be
  installed without the target package included (as Kickstart, the
  RHEL installation tool, does not support installing a specific
  version of a package~\cite{Ana11}); next, the correct package
  version must be installed; and finally, the package must be added to
  the exclude list.  If this happens out of order, then the wrong
  version of the package might be installed, or the package might not
  be installed at all.

\item Supplying a permitted update to a package is even more
  difficult, as it involves removing the package exclusion, updating
  to the correct version, and then restoring the exclusion.  A
  configuration management system would have to have tremendously
  granular control over the order in which actions are performed to
  accomplish this delicate goal.

\item As discussed earlier, Bcfg2 performs dependency resolution on
  the server side in order to provide a guarantee that a client's
  configuration is fully specified.  By using yum excludes -- which
  cannot be configured in Bcfg2's internal dependency resolver -- the
  relationship between the client and the server is broken, and Bcfg2
  will in perpetuity claim that the client is out of sync with the
  server, thus reducing the usefulness of the Bcfg2 reporting tools.

\end{itemize}

While yum excludes appear at first to be a viable option, their use to
set package versions is not replicable, consistent, and cannot be
trivially automated.

\subsection{Specifying Versions in Bcfg2}

Bcfg2 is capable of specifying specific versions of packages in the
specification, e.g.:

{\tt \small
\begin{verbatim}
<BoundPackage name="glibc" type="yum">
  <Instance version="2.13" release="1"
            arch="i686"/>
  <Instance version="2.13" release="1"
            arch="x86_64"/>
</BoundPackage>
\end{verbatim}
}

This is obviously quite verbose (more so because the example uses a
multi-arch package), and as a result of its verbosity it is also
error-prone.  Having to recopy the version, release, and architecture
of a package -- separately -- is not always a trivial process, and the
relatively few constraints of version and release strings makes it
less so.  For instance, given the package:

{\tt \small
\begin{verbatim}
iomemory-vsl-2.6.35.12-88.fc14.x86_64-
    2.3.0.281-1.0.fc14.x86_64.rpm
\end{verbatim}
}

The package name is ``iomemory-vsl-2.6.35.12-88.fc14.x86\_64'' (which
refers to the specific kernel for which it was built), the version is
``2.3.0.281'' and the release is ``1.0.fc14''.\endnote{Admittedly,
  this is a non-standard naming scheme, but no solution can be
  predicated on the idea that all RPMs are well-built.}  This can be
clarified through use of the \texttt{--queryformat} option to
\texttt{rpm}, but the fact that more advanced RPM commands are
necessary makes it clear that this approach is untenable in general.
Even more worrisome is the package epoch, a sort of ``super-version,''
which RPM cleverly hides by default, but could cause a newer package
to be installed if it was not specified properly.

Maintenance is also tedious, as it involves endlessly updating verbose
version strings; recall that a given version is just shorthand for
what we actually care about -- that a package \emph{works}.

This approach also does not abrogate the use of yum on a system to
update it beyond the appropriate point.  The only thing keeping a
package at the chosen version is Bcfg2's own self-restraint; if an
admin on a machine lacks that same self-restraint, then he or she
could easily update a package that was not to be updated, whereupon
Bcfg2 would try to downgrade it.

Finally, this approach presents specific difficulties for us, as our
adoption of Bcfg2 is far from complete; large swaths of the center
still use Cfengine 2, and some machines -- particularly compute and
storage platforms -- operate in a diskless manner and do not use
configuration management tools in a traditional manner.  They depend
entirely on their images for package versions, so specifying versions
in Bcfg2 would not help.

To clarify, using Bcfg2 forced us to reconsider this problem, and any
solution must be capable of working with Bcfg2, but it cannot be
assumed that the solution may leverage Bcfg2.

\subsection{Yum versionlock}

Using yum's own version locking system would appear to improve upon
pegging versions in Bcfg2: it works on all systems, regardless of
whether or not they use Bcfg2; and a shortcut command, \texttt{yum
  versionlock <package-name>}, is provided to make the process of
maintaining versions less error-prone.\endnote{The command in question
  merely maintains a local file on a machine, so that file would still
  have to be copied into the Bcfg2 specification, but we believe this
  would be less error-prone than copying package version details.}

It also solves many of the problems of yum excludes, but suffers from
a critical flaw in that approach: by setting package versions on the
client, the relationship between the Bcfg2 client and server would be
broken.

Combinations of these three approaches merely exhibit combinations of
their flaws.  For instance, the promising combination of yum's
versionlock plugin and specifying the version in Bcfg2 would ensure
that the Bcfg2 client and server were of a mind about package
versions, and would work on non-Bcfg2 machines; however, it would
forfeit versionlock's ease of use and require the administrator to
once again manually copy package versions.

\subsection{Spacewalk}

Spacewalk was the first full-featured solution we looked at that aims
to replace the mirroring portion of this relationship; all of the
other potential solutions listed thus far have attempted to work with
a ``dumb'' mirror and use yum features to work around the problem we
have described.  Spacewalk is a local mirror system that ``manages
software content updates for Red Hat derived [\emph{sic}] distributions''
\cite{RH10}; it is a tremendously full-featured system, with support
for custom ``channels,'' collections of packages assembled in an
ad-hoc basis.

Unfortunately, Spacewalk was a non-starter for us for the same reason
that it has failed to gain much traction in the community at large: of
the two versions of Spacewalk, only the Oracle version actually
implements all of the features; the PostgreSQL version is deeply
underfeatured, even after several years of work by the Spacewalk team
to port all of the Oracle stored procedures.

As it turns out, Red Hat has a successor in mind for Spacewalk and
Satellite: CloudForms~\cite{WS11}.  The content management portion
of CloudForms -- roughly corresponding to the mirror and repository
management functionality of Spacewalk -- is Pulp.
