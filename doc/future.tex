\section{Future Development}
\label{sec:future}

\subsection{Sponge}

At this time, Pulp is very early code; it has been in use in another
Red Hat product for a while, so certain paths are well-tested, but
other paths are pre-alpha.  Consequently, its command line interface
lacks polish, and many tasks within Pulp require extraordinary
verbosity to accomplish.  It is also not clear if Pulp is intended for
standalone use, although such is possible.

To ease management of Pulp, we have written a web frontend for
management of Pulp and its objects, called ``Sponge.'' Sponge, powered
by the Django~\cite{Dja11} web framework, provides views into the
state of Pulp repositories along with the ablity to manage its
contents. Sponge leverages Pulp's Python client API to provide
convience functions that ease our workflow.

By presenting the information visually, Sponge makes repository
management much more intuitive. Sponge extends the functionality of
Pulp by displaying the differences between a repository and its parent
in the form of a diff. These diffs give greater insight into
exactly how \texttt{stable}, \texttt{unstable}, and \texttt{live}
tiers differ. They also provide insight into the implications of a
package promotion or removal.

This is particularly important with package removal, since, as noted,
removing a package will also remove anything that requires that
specific package.  Without Sponge's diff feature and a confirmation
step, that is potentially very dangerous; Pulp itself only gives you
confirmation of the packages removed without an opportunity to confirm
or reject a removal.  The contrapositive situation -- promoting a
package pulling in unintended dependencies -- is also potentially
dangerous, albeit less so.  Sponge helps avert both dangers.

\subsection{Guaranteeing a minimum package age}

As Beattie at al. observe~\cite{BACWWS02}, the optimal time to apply
security patches is either 10 or 30 days after the patches have been
released.  Our workflow currently doesn't provide any way to guarantee
this; our weekly manual promotion of new packages merely suggests that
a patch be somewhere between 0 and 6 days old before it is promoted to
\texttt{unstable}, and 7 and 13 days old before being promoted to
\texttt{stable}.  We plan to add a feature -- either to Sponge or to
Pulp -- to promote packages only once they have aged properly.

\subsection{Other packaging formats}

In this paper we have dealt with systems using yum and RPM, but the
approach can, at least in theory, be expanded to other packaging
systems.  Pulp intends eventually to support not only Debian packages,
but actually any sort of generic content at all~\cite{Dob11-3}, making
it useful for any packaging system.  Bcfg2, for its part, already has
package drivers for a wide array of packaging systems, including APT,
Solaris packages (Blastwave- or SystemV-style), Encap, FreeBSD
packages, IPS, Mac Ports, Pacman, and Portage.  This gives a hint of
the future potential for this approach.
