\section{Background}
\label{sec:background}

It is important to understand some points about our environment, as
they provide important constraints to our solution.

We are lucky enough to run a fairly homogenous set of operating
systems consisting primarily of Red Hat Enterprise Linux and CentOS
servers, with fair numbers of Fedora and SuSE outliers.  In short, we
are dealing entirely with RPM-based packaging, and with operating
systems that are capable of using yum~\cite{Vid11}.  As yum is the
default package management utility for the majority of our servers, we
opted to use yum rather than try to switch to another package
management utility.

For configuration management, we chose to use Bcfg2~\cite{Des11} for
reasons wholly unrelated to package and software management.  Bcfg2 is
a Python and XML-based configuration management engine that ``helps
system administrators produce a consistent, reproducible, and
verifiable description of their environment''~\cite{Des11}.  It is in
particular the focus on reproducibility and verification that forced
us to consider updating and patching anew.

In order to guarantee that a given configuration -- where a
``configuration'' is defined as the set of paths, files, packages, and
so forth, that describes a single system -- is fully replicable, Bcfg2
ensures that every package specified for a system is the latest
available from that system's software repositories~\cite{JLS11}.  (As
will be noted, this can be overridden by specifying an explicit
package version.)  This grants the system administrator two important
abilities: to provision identical machines that will remain identical;
and to reprovision machines to the exact same state they were
previously in.  But it also makes it unreasonable to simply use the
vendor's software repositories (or other upstream repositories), since
all updates will be installed immediately without any vetting.  The
same problem presents itself even with a local mirror.

Bcfg2 can also use ``the client's response to the specification ... to
assess the completeness of the specification''~\cite{Des11}.  For this
to happen, the Bcfg2 server must be able to understand what a
``complete'' specification entails, and so the server does not
entirely delegate package installation to the Bcfg2 client.  Instead,
it performs package dependency resolution on the server rather than
allowing the client to set its own configuration.  This necessitates
ensuring that the Bcfg2 Packages plugin uses the same yum
configuration as the clients; Bcfg2 has support for making this rather
simple~\cite{JLS11}, but the Packages plugin does not support the full
range of yum functionality, so certain functions like the
``versionlock'' plugin and even package excludes, are not available.
Due to the architecture of Bcfg2 -- architecture designed to guarantee
replicability and verification of server configurations -- it is not
feasible or, in most cases, possible to do client-based package and
repository management.  This became critically important in selecting
a solution.
